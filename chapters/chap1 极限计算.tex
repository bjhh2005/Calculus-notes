\documentclass[../main.tex]{subfiles}

\begin{document}

\section{极限的计算}

\begin{question}
计算\[
\lim_{x \to 0} \frac{\sqrt[m]{1+\alpha x}\sqrt[n]{1+\beta x} - 1}{x}
\]
\end{question}

\begin{solution}
\begin{align*}
\text{原式}&= \lim_{x \to 0} \sqrt[m]{1+\alpha x}\frac{\sqrt[n]{1+\beta x} - 1}{x} + \lim_{x \to 0} \sqrt[n]{1+\beta x}\frac{\sqrt[m]{1+\alpha x} - 1}{x}\\
&= \frac{\alpha}{m} + \frac{\beta}{n}
\end{align*}
\end{solution}

\begin{points}
\begin{itemize}
    \item 添项减项法 / 分项求极限
    \item 幂次根号等价无穷小: $\sqrt[k]{1+u}-1 \sim u/k$
\end{itemize}
\end{points}

\begin{question}
计算
\[
\lim_{x \to 0} \frac{1 - \cos{x}\cos{x} \dots \cos{x}}{x^{2}}
\]
\end{question}

\begin{solution}
\begin{align*}
\text{原式}&= \lim_{x \to 0} \frac{(1 - \cos{x}) + (\cos{x} - \cos{x}\cos{x} \dots \cos{x})}{x^{2}}\\
&= \frac{1}{2} + \lim_{x \to 0} \frac{1 - \cos{2x} \dots \cos{x}}{x^{2}}\\
&=\cdots \quad \text{(反复使用上面的拆项方法)}\\
&=\frac{1}{2} + \frac{1}{2} \cdot 2^{2} + \frac{1}{2} \cdot 3^{2} + \dots + \frac{1}{2} \cdot n^{2}\\
&=\frac{1}{2} \cdot \frac{n(n+1)(2n+1)}{6} = \frac{n(n+1)(2n+1)}{12} 
\end{align*}
\end{solution}

\begin{points}
\begin{itemize}
    \item 连锁拆项法 / 反复添项减项
    \item $\lim \frac{1-\cos u}{u^2} = \frac{1}{2}$
    \item $\sum k^2$ 平方和公式
\end{itemize}
\end{points}

\begin{note}
前面两个都是添项减项法的运用。其实还有另一种做法。
\end{note}

\begin{question}
计算\[
\lim_{x \to 0} \frac{\sqrt[m]{1+\alpha x}\sqrt[n]{1+\beta x} - 1}{x}
\]
\end{question}

\begin{solution}
\begin{align*}
\text{原式}&=\lim_{x \to 0} \frac{\ln{[(1+\alpha x)^{\frac{1}{m}} \cdot (1+\beta x)^{\frac{1}{n}}]}}{x}\\
&= \lim_{x \to 0} \frac{\frac{1}{m}\ln{(1+\alpha x)}}{x} + \lim_{x \to 0} \frac{\frac{1}{n} \ln{(1+\beta x)}}{x}\\
&= \frac{\alpha}{m} + \frac{\beta}{n}\\
\end{align*}
\end{solution}

\begin{points}
\begin{itemize}
    \item 对数法: $u-1 \sim \ln u$ ($\text{当 } u \to 1 \text{ 时}$)
    \item 乘积极限化为和式极限 (利用 $\ln(AB)=\ln A + \ln B$)
    \item $\ln(1+u)$ 等价无穷小 ($\sim u$)
\end{itemize}
\end{points}

\begin{question}
计算\[
\lim_{x \to 0} \frac{1 - \cos{x}\cos{x} \dots \cos{x}}{x^{2}}
\]
\end{question}

\begin{solution}
\begin{align*}
\text{原式}&= -\lim_{x \to 0} \frac{\cos x \cos 2x \cdots \cos nx - 1}{x^{2}}\\
&= -\lim_{x \to 0} \frac{\ln \cos x \cdots \cos nx}{x^{2}}\\
&= -\sum_{k=1}^{n} \lim_{x \to 0} \frac{\ln \cos kx}{x^{2}}\\
&= -\sum_{k=1}^{n} \lim_{x \to 0} \frac{\cos kx - 1}{x^{2}}\\
&= -\sum_{k=1}^{n} (-\frac{1}{2}k^{2}) = \frac{n(n+1)(2n+1)}{12}\\
\end{align*}  
\end{solution}

\begin{points}
\begin{itemize}
    \item 对数法 (乘积型极限简化)
    \item $\ln(\cos u)$ 等价无穷小 ($\sim -u^2/2$)
    \item $\sum k^2$ 平方和公式
\end{itemize}
\end{points}

\begin{note}
先反用$\ln x \sim x - 1 (x \to 1)$,再正用$\ln x \sim x - 1 (x \to 1)$,尤其适用于含有"$A_1 A_2 \dots A_n - 1$"结构的极限。
\end{note}

\begin{question}
计算
\[
\lim_{x \to 0} \frac{1-\cos x \sqrt{\cos 2x} \cdots \sqrt[n]{\cos nx}}{x^{2}}
\]
\end{question}

\begin{solution}
\begin{align*}
\text{原式} &= \lim_{x \to 0} \frac{1-\cos x (\cos 2x)^{\frac{1}{2}} \cdots (\cos nx)^{\frac{1}{n}}}{x^{2}} \\
&= \lim_{x \to 0} \frac{-\ln \left( \cos x (\cos 2x)^{\frac{1}{2}} \cdots (\cos nx)^{\frac{1}{n}} \right)}{x^{2}}\\
&= -\lim_{x \to 0} \frac{\sum_{k=1}^{n} \frac{1}{k}\ln (\cos kx)}{x^{2}} \\
&= -\sum_{k=1}^{n} \lim_{x \to 0} \frac{\frac{1}{k}\ln (\cos kx)}{x^{2}} \\
&= -\sum_{k=1}^{n} \frac{1}{k} \lim_{x \to 0} \frac{-\frac{(kx)^{2}}{2}}{x^{2}} \quad \text{(利用等价无穷小 } \ln(\cos u) \sim -\frac{u^2}{2} \text{)} \\
&= \frac{n(n+1)}{4}
\end{align*}
\end{solution}

\begin{points}
\begin{itemize}
    \item 对数法 (根式/乘积混合极限)
    \item $\ln(\cos u)$ 等价无穷小 ($\sim -u^2/2$)
    \item $\sum k$ 等差数列求和公式
\end{itemize}
\end{points}

\begin{question}
计算\[
\lim_{x \to 0} \frac{(1 - \cos x) \left[x - \ln(1 + \tan x)\right]}{x^{4}}
\]
\end{question}

\begin{solution}
\begin{align*}
\text{原式}&=\lim_{x \to 0} \frac{\frac{x^2}{2} [x - \ln(1 + \tan x)]}{x^4}\\
&=\frac{1}{2} \lim_{x \to 0} \frac{x - \ln(1 + \tan x)}{x^2}\\
&=\frac{1}{2} \left[\lim_{x \to 0} \frac{x - \tan x}{x^2} + \lim_{x \to 0} \frac{\tan x - \ln(1 + \tan x)}{x^2} \right]\\
&=\frac{1}{2}\left[0+\frac{1}{2}\right]\\
&=\frac{1}{4}
\end{align*}
\end{solution}

\begin{points}
\begin{itemize}
    \item 等价无穷小替换 (乘积型)
    \item Taylor/高阶无穷小: $x - \ln(1+u) \sim u^2/2$
    \item 拆分代换 (利用 $\tan x \sim x$ 简化内部结构)
\end{itemize}
\end{points}

\begin{note}
$x- \ln (1+x) \sim \frac{x^2}{2} \quad(x \to 0)$
\end{note}

\begin{question}
计算\[
\lim_{x \to 0} \frac{(3+2\tan x)^x - 3^x}{3\sin^2 x + x^3 \cdot \cos\frac{1}{x}}
\] 
\end{question}

\begin{solution}
\begin{align*}
\text{原式}&=\lim_{x \to 0} \frac{(3+2\tan x)^x - 3^x}{3\sin^2 x}\quad (\text{无穷小的吸收律})\\
&=\lim_{x \to 0} \frac{3^{x}\left[(1+\frac{2}{3}\mathrm{tan}x)^{x}-1\right]}{3x^{2}}\\
&=\frac{1}{3} \lim_{x \to 0} \frac{x \ln(1 + \frac{2}{3} \tan x)}{x^2}\\
&=\frac{1}{3} \lim_{x \to 0} \frac{\frac{2}{3} \tan x}{x}\\
&= \frac{2}{9}
\end{align*}
\end{solution}

\begin{points}
\begin{itemize}
    \item 无穷小的吸收律 (分母简化)
    \item 指数差型极限 $A^x - B^x$ 提取公因式
    \item $(1+u)^v - 1 \sim v \ln(1+u)$
\end{itemize}
\end{points}

\begin{question}
设$f(x),g(x)$在$x=0$的邻域$U$内有定义,且对$\forall x \in U$, 均有$f(x) \neq g(x)$, 且$\lim_{x \to 0} f(x)=\lim_{x \to 0}g(x)=a>0$, 计算
$$
\lim_{x \to 0} \frac{\left[ f(x) \right]^{g(x)} - \left[ g(x) \right]^{g(x)}}{f(x) - g(x)}
$$
\end{question}

\begin{solution}
\begin{align*}
\text{原式} &= \lim_{x \to 0} [g(x)]^{g(x)} \cdot \frac{\left[\frac{f(x)}{g(x)}\right]^{g(x)} - 1}{f(x) - g(x)}\\
&= a^a \lim_{x \to 0} \frac{g(x) \cdot \ln \frac{f(x)}{g(x)}}{f(x) - g(x)}\\
&= a^{a+1} \lim_{x \to 0} \frac{\frac{f(x) - g(x)}{g(x)}}{f(x) - g(x)}\\
&= a^a
\end{align*}
\end{solution}

\begin{points}
\begin{itemize}
    \item 抽象函数极限
    \item $A^B - C^B$ 指数差型极限 (提取公因式)
    \item $\frac{f(x)}{g(x)} \to 1$ 时的关键代换: $\ln\left(\frac{f}{g}\right) = \ln\left(1 + \frac{f-g}{g}\right) \sim \frac{f-g}{g}$
\end{itemize}
\end{points}

\begin{question}
计算
\begin{align*}
\lim_{x \to 0^{+}} (e^{x} - 1 - x)^\frac{1}{\ln x}
\end{align*}
\end{question}

\begin{solution}
\begin{align*}
\text{原式}&=e^{\lim_{x \to 0^{+}} \frac{1}{\ln x} \cdot \ln(e^{x} - x-1)}\\
&=e^{\lim_{x \to 0^{+}} \frac{\ln\frac{x^2}{2}}{\ln{x}}}\\
&=e^{\lim_{x \to 0^{+}} \frac{2 \ln x - \ln 2}{\ln x}}\\
&= e^{2}
\end{align*}
\end{solution}

\begin{points}
\begin{itemize}
    \item 幂指函数极限 ($0^0$ 型)
    \item $\ln L$ 法 (转化为指数形式 $e^{\lim \text{指数} \cdot \ln\text{底数}}$)
    \item Taylor 展开 ($e^x - 1 - x \sim x^2/2$)
    \item $\frac{\ln(x^2/2)}{\ln x}$ 等价代换和洛必达
\end{itemize}
\end{points}

\begin{question}
计算
\[
\lim_{x \to 0^{+}} \frac{\ln{(\arctan{x})}}{\ln{[\ln{(1 + x)}]}}
\]
\end{question}

\begin{solution}
\[
\text{原式}= \lim_{x \to 0} \frac{\ln{x}}{\ln{x}} = 1
\]
\end{solution}

\begin{note}
这些都是等价无穷大替换,也就是如果$\alpha \sim \beta \to 0^+$,则$\ln \alpha \sim \ln \beta \to -\infty$
\end{note}

\end{document}